\section*{Introduction}
%Benoit, 3'30

\begin{frame}
	\frametitle{Introduction (1/3)}
     
    Un projet sur le traitement d'images
    \begin{itemize}
    	\item Manipuler des images matricielles
    	\item Visualiser les effets des algorithmes sur ces images
    \end{itemize}
	\begin{columns}[c]
		% 
		\column{1in}
			\begin{center}
				\includegraphics[width=1in]{Images/flou-avant.png}
			\end{center}
		\column{1in}
			\begin{center}
				\includegraphics[width=1in]{Images/flou-gaussien-3.png}
			\end{center}
	\end{columns}
	\begin{center}
		\scriptsize
			Application d'un flou sur une image
	\end{center}
\end{frame}

\begin{frame}
	\frametitle{Introduction (2/3)}
	
	\textbf{Premier objectif :} construire un environnement de développement
	\begin{itemize}
		\item Bibliothèque \bsc{ImageIn} :
		\begin{itemize}
		    \item Fournir un environnement pour la manipulation d'images \\
			$\longrightarrow$ Lecture/écriture sous de multiples formats \\
			$\longrightarrow$ Créer un ensemble d'algorithmes \\
		    \item Faciliter le développement d'un nouvel algorithme de traitement
		\end{itemize}
		\item Interface générique :
		\begin{itemize}
		    \item Permettre le développement d'une application associée
		\end{itemize}
	\end{itemize}
\end{frame}

\begin{frame}
	\frametitle{Introduction (3/3)}
	
	\textbf{Second objectif :} construire des applications pédagogiques\\

	\begin{itemize}
		\item Montrer les possibilités de l'environnement
		\item Tester facilement et rapidement nos algorithmes
	\end{itemize}
\end{frame}
