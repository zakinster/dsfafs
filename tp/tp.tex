\documentclass[12pt,a4paper]{article}

%le préambule
%pour taper son texte directement avec des lettres accentuées,
%     donc é plutôt que \'e par exemple
% que la césure s'effectue correctement
% que lors de la création d'un pdf, le codage permette de
%     copier-coller du texte d'un pdf vers le bloc note
%     parfaitement.
\usepackage[T1]{fontenc}
\usepackage[utf8]{inputenc}
\usepackage[french]{babel}

\usepackage{amsmath}
\usepackage{amsfonts}
\usepackage{amssymb}
\usepackage{graphicx}
\usepackage[left=2cm,right=2cm,top=2cm,bottom=2cm]{geometry}
\usepackage{listings}


\title{TP Administration Serveur - Apache/Tomcat}

%document principal
\begin{document}

\maketitle{}

\section{Introduction}
\paragraph{}
L'objectif du TP est de découvrir un ensemble assez large d'opérations d'administration sous Debian à travers la tâche classique d'installation d'un serveur web. Les logiciels choisis pour cet exemple sont Apache httpd (plus connu sous le simple nom d'Apache) et Apache Tomcat, souvent simplement appelé Tomcat.

\paragraph{}
En dehors des configurations spécifiques à ces deux programmes, la plupart des connaissances acquises dans le cours et le TP pourront resservir pour d'autres logiciels comme Squid (serveur proxy), OpenVPN, ou d'autres cas plus particuliers (serveur DC ou serveur Minecraft par exemple).

\paragraph{}
Certaines étapes du TP ne sont volontairement pas trop détaillées. L'objectif est d'apprendre à utiliser les pages de manuel (commande man) et la documentation des logiciels. Cependant, n'hésitez pas à poser des questions si vous êtes bloqués !

\subsection*{liens utiles}
    
\begin{itemize}
\item Documentation d'Apache httpd : \textit{http://httpd.apache.org/docs/2.2/}
\item Documentation d'Apache Tomcat : \textit{http://tomcat.apache.org/tomcat-7.0-doc/index.html}
\item Page de téléchargement de Tomcat : \textit{http://tomcat.apache.org/download-70.cgi}
\end{itemize}

\subsection*{Environnement de travail}
\paragraph{}
Afin d'être dans des conditions les plus réalistes possibles, vous allez travailler sur un serveur dédié (virtuel) hébergé chez ovh. Vous disposez d'un serveur et de l'accès root correspondant (demandez nous le mot de passe root !).

\paragraph{}
Chaque serveur est pointé par deux noms de domaines : a.bx.3f0.net et b.bx.3f0.net, où "x" est le numéro de votre binome (de 1 à 8). Connectez vous avec ssh à votre serveur et démarrez le TP !

%%%%%%%%%%%%%%%%%%%%%%%%%%%%%%%%%%%ù
\section{Installation et configuration d'Apache}
\subsection{Installation}

\paragraph{Installez Apache httpd\\}
Apache httpd est un logiciel très répandu et est donc disponible dans les dépots Debian. A moins d'avoir besoin d'une configuration très particulière nécéssitant de compiler Apache à la main, c'est la meilleure manière de l'installer.

\begin{lstlisting}
$ apt-get install apache2
\end{lstlisting}

\paragraph{Vérifiez que Apache fonctionne\\}
Apache est installé et lancé avec une configuration basique. Vous pouvez vous en convaincre en listant les services qui écoutent sur le serveur : 

\begin{lstlisting}
$ netstat -a
\end{lstlisting}

\paragraph{}
Vous devriez avoir un service écoutant sur le port 80, correspondant au service http. Tapez l'adresse du serveur dans votre navigateur. Vous devriez voir la page par défaut d'Apache (disponible dans /var/www).

\subsection{Configuration de deux hôtes virtuels}
\paragraph{}
Par défaut, apache est configuré avec un seul hôte, dont la configuration se trouve dans le fichier /etc/apache2/sites-availables/default. La racine par défaut du serveur est /var/www.

\paragraph{Créez les répertoires racines de chacun des sites web\\}
Il est nécéssaire de créer un répertoire racine pour chacun des hôtes virtuels. Une pratique courante est d'en faire des sous-dossiers de /var/www. Créez un fichier index.html dans chacun de ces dossiers, avec un signe distinctif qui vous permettra de vérifier le bon fonctionnement de votre configuration.

\paragraph{Créez les deux hôtes virtuels\\}
Ces deux hôtes virtuels doivent correspondre aux deux noms de domaines pointant sur votre serveur. Il est conseillé de copier le fichier default pour les deux nouveaux hôtes virtuels et de modifier/ajouter seulement les options intéressantes. Desactivez ensuite l'hôte par défaut.

\paragraph{}
Une fois les sites créés et activés, testez la configuration en accédant à votre serveur depuis un navigateur en utilisant les deux noms de domaines différents.

\paragraph{}
Documentation intéressante : 
\begin{itemize}
\item Le manuel de la commande a2ensite/a2dissite
\item La documentation des options de configuration ServerName, DocumentRoot, et des virtualhosts en général
\end{itemize}

\subsection{Exemple de configuration avancée - Utilisation d'un Handler}

\subsubsection{Du contenu statique au contenu dynamique}

\paragraph{}
Maintenant que le serveur HTTP Apache est installé et que la gestion de contenu statique est fonctionnelle, il est possible de s'intéresser à la gestion de contenu dynamique. Pour ce faire, il existe plusieurs méthodes qui permettent d'interfacer Apache avec du code externe capable d'interpréter des pages dynamiques. Nous allons ici nous intéresser à la plus simple d'entre elles : le CGI.

\subsubsection{Un handler CGI}

\paragraph{}
L'objectif est de déléguer la gestion de certains types de fichiers à un exécutable externe à Apache. La première étape consiste donc à obtenir cet exécutable, que l'on appelera \textit{Handler}. Vous disposez dans les fichiers du TP du fichier source d'un handler simpliste développé en C++. Afin qu'Apache puisse executer ce programme, vous devez le compiler puis placer l'executable dans le dossier $/usr/lib/cgi-bin/$.

\subsubsection{Configuration du handler}
\paragraph{}
Pour la suite de cette partie, la configuration se fera sur un des VirtualHost définis préalablement.

\paragraph{}
Pour déléguer la gestion de certain types de fichiers vous devez commencer par définir les extensions de fichiers que vous voulez prendre en compte en utilisant la commande AddHandler.

\begin{lstlisting}
AddHandler app-binary .c 
\end{lstlisting}

Vous devez ensuite définir le handler que vous voulez appeler en utilisant la commande Action.

\begin{lstlisting}
Action app-binary /cgi-bin/handler
\end{lstlisting}

%%%%%%%%%%%%%%%%%%%%%%%%%%%%%%%%%%%%%%%%%%ùù
\section{Installation et configuration de Tomcat}

\subsection{Installation et lancement}

\paragraph{}
Tomcat est également disponible dans les dépots Debian, cependant la dernière version (7.0) n'est pas encore disponible pour la version de Debian actuelle. Vous allez donc devoir télécharger tomcat sur le site web d'apache et l'installer à la main.

\paragraph{Installez java via le gestionnaire de paquets\\}
Le nom du paquet qui nous intéresse est sun-java6-jdk

\paragraph{Téléchargez et décompressez tomcat\\}
L'adresse de téléchargement est disponible dans les liens utiles au début du TP. En général, les logiciels installés à la main sont installés dans le dossier /opt. Vous pouvez également l'installer ailleurs, mais il vous faudra adapter certaines étapes de la suite du TP.

\paragraph{}
Documentation intéressante : 
\begin{itemize}
\item Le manuel de la commande wget
\item La manuel de la commande tar
\end{itemize}

\paragraph{Lancez Tomcat\\}
La configuration par défaut de Tomcat nous conviendra bien. Il s'agit maintenant de le lancer. Comme nous ne l'avons pas installé via le gestionnaire de paquets, il n'y a pas de script de lancement automatique dans /etc/init.d. Il serait possible d'en créer un mais nous n'en parlerons pas dans ce TP. Les instructions de lancement se trouvent dans le fichier RUNNING.txt de l'archive de tomcat (sections 3 et 4). Pour l'emplacement de la jdk, cherchez du coté de /usr/lib/jvm.

\paragraph{}
Une fois Tomcat lancé, testez son bon fonctionnement en vous connectant à votre serveur sur le port 8080 depuis votre navigateur (http://a.bx.3f0.net:8080/).

\paragraph{Configurez les droits pour pouvoir accéder au manager\\}
La page d'accueil de Tomcat comporte des liens d'administration du serveur (les trois boutons sur la droite). Cependant, aucun accès n'est configuré pour ces pages. Vous devrez modifier la configuration de Tomcat dans le fichier conf/tomcat-users.xml et ajouter un utilisateur ayant le role "manager-gui" pour pouvoir accéder à ces pages de configuration. N'oubliez pas de redémarrer Tomcat avant de tester !

\section{Liaison Apache/Tomcat}

\paragraph{}
Comme vous avez pu le constater, Tomcat écoute par défaut sur le port 8080. Il n'est pas possible de le faire écouter sur le port 80 car Apache écoute déjà sur ce port. Pour accéder à Tomcat sans ajouter :8080 à la fin du nom de domaine, il est possible de configurer Apache pour qu'il redirige certaines requêtes http à tomcat.

\paragraph{}
Il serait possible d'utiliser un handler comme nous l'avons fait pour le C++, mais cette solution serait complexe et inutile puisqu'il existe un module apache permettant de faire cette liaison : mod\_jk.

\paragraph{Installez le connecteur mod\_jk\\}
Le nom du paquet est "libapache2-mod-jk". 

\paragraph{Configurez le worker correspondant à tomcat\\}
La configuration se fait dans le fichier /etc/libapache2-mod-jk/workers.properties. Les options intéressantes sont en particulier la localisation de votre installation de Tomcat et la localisation de votre JRE. Relevez le nom du worker, il sera utile plus tard.

\paragraph{}
Vous devez ensuite configurer l'emplacement du fichier de configuration dans le fichier /etc/apache2/mods-available/jk.load avec l'option de configuration JkWorkersFile.

\paragraph{Configurez l'un de vos virtualhosts pour rediriger toutes les requêtes vers Tomcat\\}
Choisissez l'un de vos virtualhosts et modifiez son fichier de configuration.

\paragraph{}
Documentation utile : 
\begin{itemize}
\item Documentation de l'option JkMount d'Apache
\item Documentation du mod\_jk en général
\end{itemize}

\paragraph{Testez votre configuration en accédant au virtualhost que vous venez de configurer\\}
Vous devriez avoir accès à la page d'accueil de Tomcat en vous rendant sur le virtualhost que vous venez de configurer \textbf{sans} préciser le numéro de port. La requête doit bien être recue par Apache, puis transmise à Tomcat.

\paragraph{Optionnel : Déployez une application Java EE\\}
Vous pouvez maintenant déployer une application Java EE via l'interface de tomcat, et y accéder via apache (http://a.bx.3f0.net/application). Si vous voulez y accéder directement, sans préciser le nom de l'application, il faudra modifier la configuration d'Apache, en utilisant l'url rewriting par exemple.

\section{Pour les plus rapides}

\paragraph{}
Si vous souhaitez tester d'autres choses, voici une liste d'idées de tâches d'administration.
\begin{itemize}
\item Installer PHP et créer une page web dynamique
\item Installer MySQL et PhpMyAdmin
\item Utiliser des fichiers .htaccess pour restreindre l'accès à certaines pages web
\item Installer un serveur ftp
\item Installer un serveur proxy
\item ...
\end{itemize}

\end{document}
