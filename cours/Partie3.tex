\section{Démarche d'utilisation}
%Thomas, 4'

\subsection{Mise en situation}
\begin{frame}
  Eudes Euzie a imaginé un algorithme révolutionnaire et désire pouvoir en faire
  profiter ses pairs, il choisit \bsc{Detiq-T}~:

  \begin{itemize}
    \item Implémentation facilitée de l'algorithme (\bsc{ImageIn})
    \item Création rapide d'une interface graphique (\bsc{GenericInterface})
  \end{itemize}
\end{frame}

\subsection{Utiliser \bsc{ImageIn}}
\begin{frame}{Utiliser \bsc{ImageIn} (1/2)}
  Qu'est-ce qu'un algorithme~?

  \begin{itemize}
    \item Pour le traitement d'image~: une fonction
    \item Pour \bsc{ImageIn}~: un foncteur\footnote{Foncteur = objet pouvant se
      comporter comme une fonction}
  \end{itemize}

  Ce que l'on veut obtenir~:

  \begin{itemize}
    \item Création~: \texttt{MonAlgorithme algo(3, 5);}
    \item Utilisation~: \texttt{algo(monImage);}
  \end{itemize}
\end{frame}

\begin{frame}{Utiliser \bsc{ImageIn} (2/2)}
  Réalisation~: hériter de la classe \texttt{Algorithm\_t<I, A>}

  \begin{itemize}
    \item Choisir les paramètres templates
      \begin{itemize}
        \item \texttt{I}~: le type d'image de retour
        \item \texttt{A}~: l'arité de l'algorithme
      \end{itemize}
    \item Implémenter la fonction \texttt{algorithm(vector<Image>)}
  \end{itemize}

  Et c'est tout~!
\end{frame}

\subsection{Utiliser \bsc{GenericInterface}}
\begin{frame}{Utiliser \bsc{GenericInterface} (1/3)}
  Implémenter un algorithme ne suffit pas, il faut~:

  \begin{itemize}
  	\item Le tester
	\item Pouvoir le montrer
  \end{itemize}

  L'interface générique de \bsc{Detiq-T} permet de le faire.
\end{frame}

\begin{frame}{Utiliser \bsc{GenericInterface} (2/3)}

  \begin{center}
    Une fonctionnalité $\Rightarrow$ un service
  \end{center}

  Héritage de deux classes~: 

  \begin{itemize}
    \item \texttt{Service}~: solution générale
    \item \texttt{AlgorithmService}~: pratique pour les cas les plus simples
  \end{itemize}
\end{frame}

\begin{frame}[containsverbatim]{Utiliser \bsc{GenericInterface} (3/3)}
L'interface \texttt{Service}~:

\begin{minted}{cpp}
class Service
{
public:
  /* modifier l'interface pour rendre le Service
     utilisable */
  virtual void display (GenericInterface* gi) = 0;
  /* met en place la gestion des evenements */
  virtual void connect (GenericInterface* gi) = 0;
};
\end{minted}
%  Dans le cas d'\texttt{AlgorithmService} (cas simples)~:

%  \begin{itemize}
%    \item L'appel de \texttt{applyAlgorithm(Algorithm\_t)} peut suffire.
%  \end{itemize}
\end{frame}


\begin{frame}[containsverbatim]
Avec \texttt{AlgorithmService}, le reste se limite à une unique fonction~:

\begin{minted}{cpp}
void applyAlgorithm(Algorithm_t<Image>* algo)
\end{minted}

\vspace{1cm}

\textbf{Exemple :} application Diphtering

\begin{minted}{cpp}
void DitheringService::applyDithering()
{ 
    Dithering* algo = new Dithering;
    applyAlgorithm(algo);
}
\end{minted}

$\longrightarrow$ trois fonctions à implémenter
\end{frame}
