\section{Tomcat : conteneur de servlet Java EE}
%Nicolas
\subsection{Presentation}
\begin{frame}
  \frametitle{Présentation de Tomcat}

  \begin{itemize}
    \item Projet open-source de la fondation Apache
    \item Serveur d'applications Java
    \begin{itemize}
      \item Servlet (génération de page Html, xml etc...)
      \item JavaServer Pages
    \end{itemize}
    \item spécifications du Java Community Process
  \end{itemize}
  \includegraphics[scale=0.5]{Images/marcheJEE.png}
\end{frame}

\begin{frame}
  \frametitle{Autres serveurs d'applications existants}
En réalité, Tomcat est seulement un conteneur de servlets, il existe des serveurs implémentant les normes JEE (sécurité, EJB, transactions globales...) :
\begin{itemize} 
  \item GlassFish (Développé par Oracle)
  \item jBoss (Utilise Tomcat pour la partie Servlet/JSP)
  \item Websphere (IBM)
  \item ...
\end{itemize}
\end{frame}

\begin{frame}
  \frametitle{Présentation Tomcat}
  \includegraphics[scale=0.5]{Images/serveurapp.png}
\end{frame}

\subsection{Configuration}
\begin{frame}
  \frametitle{Fichiers de configuration}

		Répertoire d'installation : CATALINA\_HOME
\begin{itemize} 
    \item \textbf{bin/} exécutables de Tomcat(démarrage du serveur)
    \item \textbf{common/} classes et librairies partagées par les applications web
    \item \textbf{conf/} Fichiers de configuration
    \item \textbf{Logs/} Logs d'accès, erreurs...
    \item \textbf{server/} contient les applications web du serveur en lui-même
    \item \textbf{shared/} contient les classes et librairies partagées par toutes les applications web hébergées sur le serveur
    \item \textbf{temp/}
    \item \textbf{webapps/} contient les applications web
    \item \textbf{work/} contient les JSP compilées
\end{itemize} 
\end{frame}


\begin{frame}[fragile]
  \frametitle{conf/server.xml}

\begin{itemize}
  \item \$CATALINA\_HOME/server.xml : principal fichier de configuration
  \item Eléments conteneurs (obligatoires) : Engine, Host, Connector
  \item Définition des services, protocoles, port utilisés, redirection vers des ressources externes, des serveurs externes
  \item Eléments facultatifs : GlobalNamingResources, Resources, Realm et Valve.
\end{itemize} 

\begin{lstlisting}
  <Server port="8009" shutdown="SHUTDOWN">
  ...
  </Server>
\end{lstlisting}
		
\end{frame}


\begin{frame}[fragile]
  \frametitle{conf/server.xml}
		Elément Connecteur
		\begin{itemize} 
   		\item objet Java capable d'écouter un port précis et comprenant un protocole précis
    	\item redirige les requêtes qu'il reçoit au moteur de servlets
      \item C'est l'élément qui reçoit les requêtes HTTP et les transmet au conteneur de servlet
		\end{itemize}

        \begin{lstlisting}
<Connector port="8080"
    protocol="HTTP/1.1"
    connectionTimeout="20000"
    maxThread="100"
    maxCount="100"
    redirectPort="8443" />
        \end{lstlisting}
\end{frame}

\begin{frame}[fragile]
    \frametitle{conf/server.xml}
		Eléments Engine et Host
		\begin{itemize}
   		\item \textbf{Elément Engine} : modélise le moteur de servlet, contient un ou plusieurs hosts
    	\item \textbf{Elément Host} hôte virtuel (Similaire aux virtualhosts d'Apache)
		\end{itemize}

        \begin{lstlisting}
<Engine name="Catalina">
  <Host name="localhost" appBase="webapps"
    unpackWARs="true" autoDeploy="true">
    <!-- contenu de l'element Host -->
  </Host>
</Engine>
        \end{lstlisting}
\end{frame}

\begin{frame}[fragile]
  \frametitle{conf/server.xml}
		Resssources et variables d'environnement

        \begin{lstlisting}
<GlobalNamingResources>
  <Resource name="UserDatabase" auth="Container" type="org.apache.catalina.UserDatabase" description="User database that can be updated and saved" factory="org.apache.catalina.users.MemoryUserDatabaseFactory" pathname="conf/tomcat-users.xml" />
  <Environment name="maxRetry" type="java.lang.Integer" value="10" override="false"/>
</GlobalNamingResources>

        \end{lstlisting}
\end{frame}

\subsection{Déploiement d'une application}
\begin{frame}
\frametitle{Via le manager}
	Installation d'une application web sous la forme d'une archive war
  \includegraphics[scale=0.3]{Images/tomcat-deploy-war.png}
\end{frame}

\begin{frame}
  \frametitle{Directement dans le dossier de Tomcat}

  \begin{itemize}
    \item Placer l'archive war dans le dossier webapps/
    \item Tomcat décompresse l'application au démarrage
    \item L'application est accessible par l'url http://serveur.tld/nom\_archive\_war
  \end{itemize}
\end{frame}

