\section{Généralités sur l'administration sous Linux}
%Florian, 4'

\subsection{Objectifs du projet}
\begin{frame}{Objectifs du projet}

	Création d'un environnement de développement :
	
	\begin{itemize}
		\item Une bibliothèque dédiée à la représentation des images et à leurs traitements
		\item Un environnement graphique pour les applications
		\item Des applications pédagogiques basées sur ces outils
	\end{itemize}

	3 objectifs :
	
	\begin{itemize}
		\item Portabilité : C++, bibliothèques multiplateformes
		\item Extensibilité : conception souple
		\item Open-source : code LGPL, bibliothèques libres
	\end{itemize}
	
\end{frame}

\subsection{ImageIn}
\begin{frame}{Bibliothèque \bsc{ImageIn}}

	Bibliothèque de traitement d'images matricielles :
	
	\begin{itemize}
		\item Chargement et sauvegarde de multiples formats (PNG, JPEG et BMP)
		\item Images de différentes profondeurs, nombre de canaux variable
		\item Algorithmes (binarisation, filtrage, ...)
		\item Outils (histogrammes, ...)
	\end{itemize}
	
\end{frame}

\subsection{Interface générique}
\begin{frame}{Interface générique}
	
	\textbf{Objectif :} Minimiser le temps de développement d'une nouvelle application
	
	\begin{itemize}
		\item Personnalisable facilement pour appliquer différents algorithmes
		\item Fournissant les outils de base de l'analyse d'images (grille de pixels, histogrammes)
	\end{itemize}

	
\end{frame}

\begin{frame}{Interface générique (aperçu)}
	\begin{center}
		\includegraphics[scale=0.23]{Images/geni.png} \\
		\tiny Aperçu de l'interface générique
	\end{center}	
\end{frame}

\subsection{Applications pédagogiques}
\begin{frame}{Ensemble d'applications}

	Des applications pédagogiques pour se familiariser avec le fonctionnement des algorithmes :
	
	\begin{itemize}
		\item Visualisation des plans de bits
		\item Filtrage
		\item Binarisation
		\item Étiquetage des composantes connexes
		\item Morphologie mathématique
	\end{itemize}

\end{frame}
