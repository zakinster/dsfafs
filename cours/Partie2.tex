\section{Architecture et conception}

\subsection{Bibliothèque ImageIn}
%Sacha, 4'

\begin{frame}
	\frametitle{Une bibliothèque de traitement d'images}
	\begin{center}
	\includegraphics[scale=0.58]{couches.pdf}
	\end{center}
\end{frame}

\begin{frame}
	\frametitle{Une représentation générique et extensible des images}
	\begin{itemize}
		\item Une classe générique : \texttt{Image\_t}
		    \begin{itemize}
			\item Généralisation de la profondeur par template
		    \end{itemize}
		    \small
		    ~~~$\longrightarrow$ Permet de représenter tout type d'images\\
		    ~~~$\longrightarrow$ Permet de faire des algorithmes génériques
		    \normalsize
        ~\\~\\
		\item Des classes spécialisées : \texttt{RgbImage\_t}, \texttt{GrayscaleImage\_t}
		    \begin{itemize}
		    \item Spécialisation par héritage de \texttt{Image\_t}
		    \end{itemize}
		    \small
		     ~~~$\longrightarrow$ Simplifie l'interface d'utilisation\\
		     ~~~$\longrightarrow$ Permet de typer les algorithmes
		    \normalsize
	\end{itemize}
\end{frame}

\begin{frame}
	\frametitle{Une représentation générique et extensible des algorithmes}
	\begin{itemize}
		\item Un classe abstraite générique : \texttt{GenericAlgorithm\_t}
		    \begin{itemize}
			\item Généralisation de l'arité par template
		    \end{itemize}
		    \small
		    ~~~$\longrightarrow$ Augmente l'expressivité\\
		    ~~~$\longrightarrow$ Simplifie l'interface d'utilisation
		    \normalsize
		\item Une classe abstraite spécifique : \texttt{Algorithm\_t}\\
		    \begin{itemize}
		    \item Spécialisation par héritage de \texttt{GenericAlgorithm\_t}
		    \item Typage de l'image de retour par template
		    \end{itemize}
       \tiny ~\\ \normalsize
    \item Simple à mettre en oeuvre pour l'utilisateur
        \begin{itemize}
            \item Une seule méthode à implémenter
        \end{itemize}
	\end{itemize}
\end{frame}

\begin{frame}
	\frametitle{Fonctionnalités supplémentaires}
	Un ensemble d'outils :
	\begin{itemize}
  
		\item Calcul d'histogrammes et d'histogrammes de projection
		\item Algorithmes utilitaires de traitement
		\begin{itemize}
		    \item Opérations pixels par pixels ($+,-,\times,/,\sim,\&,|$), distances, etc.
		\end{itemize}
	\end{itemize}
  ~\\

	Un environnement de test automatique :
	\begin{itemize}
	    \item Cadre de développement pour tester les algorithmes 
	\end{itemize}
\end{frame}

\subsection{Interface générique}
%Samuel, 4'

\begin{frame}
	\frametitle{La base des applications pédagogiques}
	Applications unitaires~:
	\begin{itemize}
		\item Un algorithme par application
	\end{itemize}

\vspace{0.6cm}

	Interface générique :
	\begin{itemize}
		\item Ergonomie commune pour les applications
		\item Facilité de développement
		\item Liberté d'enrichissement et de personnalisation
	\end{itemize}
\end{frame}

\begin{frame}
	\frametitle{Des services pour contrôler l'interface}
	\begin{itemize}
		\item Services de base pour un fonctionnement minimal
			\begin{itemize}
				\item Gestion ouverture/sauvegarde
				\item Gestion des images ouvertes
				\item Gestion menus contextuels
			\end{itemize}
		\item Un service dédié aux algorithmes
		\item Ajout de services pour personnaliser
	\end{itemize}
\end{frame}

\begin{frame}
	\frametitle{Des widgets pour agrémenter l'utilisation}
	\begin{itemize}
		\item Widgets de base assurant une ergonomie commune
			\begin{itemize}
				\item Affichage des images
				\item Affichage d'histogrammes (normaux, projections, etc.)
				\item Une grille de pixels
				\item Une barre de navigation 
			\end{itemize}
	\end{itemize}
	\begin{center}
		\includegraphics[scale=0.25]{Images/nav.png}~~~~
		\includegraphics[scale=0.25]{Images/histo.png}
	\end{center}


\end{frame}

\begin{frame}
	\begin{itemize}
		\item Widgets réutilisables
	\end{itemize}
	\begin{center}
		\includegraphics[scale=0.3]{Images/bit.png}\\
		\scriptsize Exemple du widget de BitPlane, une de nos applications
	\end{center}
\end{frame}
	
\begin{frame}
	\begin{itemize}
		\item Chaque application peut avoir ses propres widgets
	\end{itemize}
	\begin{center}
		\includegraphics[scale=0.28]{Images/bina.png}\\
		\scriptsize Exemple du widget de Binarisation, une de nos applications
	\end{center}
\end{frame}
